%% This work is licensed under a Creative Commons
%% Attribution-ShareAlike 3.0 Unported License.
%% http://creativecommons.org/licenses/by-sa/3.0/

%% Copyright 2007-2013  Ralf Hemmecke <ralf@hemmecke.org>

%% This material has initially been prepared for the
%% Axiom Workshop 2007 (June 14-16, 2007, Hagenberg, Austria)
%% http://www.risc.jku.at/conferences/summer2007/

%--rhx: Use 'make beamer' or 'make article' in order to compile this
%--rhx: file. The MAKE command will provide an appropriate 
%--rhx: '\documentclass' line, because we want to make an article or a
%--rhx: beamer presentation out of this file.

\mode<presentation>
{
  \usetheme{Warsaw}
  \setbeamertemplate{navigation symbols}{}
  \setbeamercovered{transparent}
  % or whatever (possibly just delete it)
}
\mode<article>
{
  \usepackage{hyperref}
  \usepackage{fullpage}
}

\usepackage[english]{babel}
\usepackage[latin1]{inputenc}
\usepackage{times}
\usepackage[T1]{fontenc}

\title{Introduction to Aldor}

\author[Ralf Hemmecke]%
       {Ralf Hemmecke \\\footnotesize{\texttt{<ralf@hemmecke.org>}}}
\date{14-Jun-2007\\
  \footnotesize{This work is licensed under CC-BY-SA-3.0.\\
    \url{http://creativecommons.org/licenses/by-sa/3.0/}}%
}

\subject{Introduction to Aldor}

%%%%%%%%%%%%%%%%%%%%%%%%%%%%%%%%%%%%%%%%%%%%%%%%%%%%%%%%%%%%%%%%%%%
\usepackage{listings}
\lstset{language={}}
\lstnewenvironment{myverbatim}%
  {\lstset{delim=[il][\color{blue}]&}\footnotesize\ttfamily}%
  {}
\lstnewenvironment{myverbatim2}%
  {\lstset{delim=[is][\color{red}]&&}\footnotesize\ttfamily}%
  {}
\usepackage{verbatim}
\newenvironment{smallverbatim}%
   {\footnotesize\verbatim}%
   {\endverbatim}

%%%%%%%%%%%%%%%%%%%%%%%%%%%%%%%%%%%%%%%%%%%%%%%%%%%%%%%%%%%%%%%%%%%
%%%%%%%%%%%%%%%%%%%%%%%%%%%%%%%%%%%%%%%%%%%%%%%%%%%%%%%%%%%%%%%%%%%
%%%%%%%%%%%%%%%%%%%%%%%%%%%%%%%%%%%%%%%%%%%%%%%%%%%%%%%%%%%%%%%%%%%
\begin{document}

\maketitle
\begin{frame}<presentation>
  \titlepage
\end{frame}

\begin{frame}
  \frametitle<presentatio>{Outline}
  \tableofcontents
\end{frame}

%%%%%%%%%%%%%%%%%%%%%%%%%%%%%%%%%%%%%%%%%%%%%%%%%%%%%%%%%%%%%%%%%%%
%%%%%%%%%%%%%%%%%%%%%%%%%%%%%%%%%%%%%%%%%%%%%%%%%%%%%%%%%%%%%%%%%%%
\section{General remarks}
%%%%%%%%%%%%%%%%%%%%%%%%%%%%%%%%%%%%%%%%%%%%%%%%%%%%%%%%%%%%%%%%%%%
%%%%%%%%%%%%%%%%%%%%%%%%%%%%%%%%%%%%%%%%%%%%%%%%%%%%%%%%%%%%%%%%%%%

%%%%%%%%%%%%%%%%%%%%%%%%%%%%%%%%%%%%%%%%%%%%%%%%%%%%%%%%%%%%%%%%%%%
\subsection{Historical Remarks}
%%%%%%%%%%%%%%%%%%%%%%%%%%%%%%%%%%%%%%%%%%%%%%%%%%%%%%%%%%%%%%%%%%%

Aldor has its origin in the Computer Algebra System (CAS) Scratchpad
developed by IBM.
%
While Scratchpad I (1971) was in concept similar to Reduce, its
successor Scratchpad II (1977) introduced types to computer algebra.
The implementation of Scratchpad II started in 1982 at IBM.
%
In 1991 Scratchpad II was sold to NAG and renamed to Axiom.

Axiom provides a language in order to use and extend its capabilities.
In fact, there are now two languages supported by Axiom. These
languages are mostly identical, but they differ in a few minor
details.

The first language is the command language that Axiom understands at
its input prompt. One can collect such commands in \texttt{.input}
files and use \texttt{)read} in order to read them into Axiom.
%
This language is mainly targetted for an interpreter session.

The second language is called SPAD (abbreviation of ScratchPAD) and is
used to write extensions of the Axiom library. There is a SPAD
compiler that compiles \texttt{.spad} files into object files.

Axiom is different from other CAS, since it has no concept of kernel
features written in a highly optimized way (for example written in C)
and extension features (written in an interpreted language). Rather
Axiom's Algebra library is completely written in SPAD and any
extension of the Algebra code runs at the same (compiled) speed as the
Algebra code itself.

However, a main disadvantage of Axiom was that its language had never
been rigorously defined. That became the birth of Aldor. Aldor,
started under the name $\mathrm{A}^\sharp$ at IBM and was later renamed to
AxiomXL (Axiom extension language). Now it is know under the name
Aldor and hosted at \url{http://www.aldor.org}.




%%%%%%%%%%%%%%%%%%%%%%%%%%%%%%%%%%%%%%%%%%%%%%%%%%%%%%%%%%%%%%%%%%%
\begin{frame}<presentation>
  \frametitle{Origin of Aldor}
  \begin{itemize}
  \item ScratchPad I (1971) %Jenks
    % see \cite[p. 21]{Doye:PhD:1997}
  \item ScratchPad II (1977, %Jenks, then Davenport and Barton
    1982) %implementation started
  \item Axiom (1991)
  \item $\mathrm{A}^\sharp$ $\longrightarrow$ AxiomXL $\longrightarrow$ Aldor
  \item IBM $\longrightarrow$ NAG $\longrightarrow$ \texttt{www.aldor.org}
  \end{itemize}
\end{frame}













%%%%%%%%%%%%%%%%%%%%%%%%%%%%%%%%%%%%%%%%%%%%%%%%%%%%%%%%%%%%%%%%%%%
\subsection{What is Aldor?}
%%%%%%%%%%%%%%%%%%%%%%%%%%%%%%%%%%%%%%%%%%%%%%%%%%%%%%%%%%%%%%%%%%%

%%%%%%%%%%%%%%%%%%%%%%%%%%%%%%%%%%%%%%%%%%%%%%%%%%%%%%%%%%%%%%%%%%%
\begin{frame}
  \frametitle{\textbf{A} \textbf{L}anguage for \textbf{D}escribing
    \textbf{O}bjects and \textbf{R}elations}
  \begin{itemize}
  \item Aldor is an ordinary programming language.
  \item design goals: generality, composability, efficiency
  \item combines object-oriented and functional programming styles 
  \item emphasis on uniform handling of functions and types,\\
    types and functions are \alert{first class}, i.e.:
    \begin{itemize}
    \item usable in function arguments
    \item constructable at runtime
    \end{itemize}
  \end{itemize}
\end{frame}








%%%%%%%%%%%%%%%%%%%%%%%%%%%%%%%%%%%%%%%%%%%%%%%%%%%%%%%%%%%%%%%%%%%
\begin{frame}
  \frametitle{Distinguished Features}
  \begin{itemize}
  \item parametric polymorphism, dependent types
  \item post facto type extensions
  \item conditional exports
  \end{itemize}
\end{frame}








%%%%%%%%%%%%%%%%%%%%%%%%%%%%%%%%%%%%%%%%%%%%%%%%%%%%%%%%%%%%%%%%%%%
\begin{frame}
  \frametitle{Aldor Compiler}

  Aldor compiler can produce
  \begin{itemize}
  \item stand-alone executable programs
  \item object libraries in native OS formats
  \item portable byte code
  \item C and LISP code
  \end{itemize}

  Aldor code can be linked with C and Fortran code.
\end{frame}








%%%%%%%%%%%%%%%%%%%%%%%%%%%%%%%%%%%%%%%%%%%%%%%%%%%%%%%%%%%%%%%%%%%
\begin{frame}
  \frametitle{Aldor Distribution}

  Distribution from aldor.org includes
  \begin{itemize}
  \item compiler for Aldor language
  \item interpreted, interactive environment for the Aldor language
  \item libaldor and libalgebra libraries
  \end{itemize}
\end{frame}














%%%%%%%%%%%%%%%%%%%%%%%%%%%%%%%%%%%%%%%%%%%%%%%%%%%%%%%%%%%%%%%%%%%
\begin{frame}[fragile]
  \frametitle{Aldor Properties}

  \begin{itemize}
  \item Aldor is a statically typed language
  \item everything has a type
    \begin{enumerate}
    \item 5 is of type \texttt{Integer}
    \item \texttt{Integer} is of type \texttt{IntegerType}
    \item \texttt{IntegerType} is of type \texttt{Category}
    \end{enumerate}
  \item 3 type levels
    \begin{enumerate}
    \item object level (variables, functions)
    \item domain and package level
    \item category level
    \end{enumerate}
  \end{itemize}
\end{frame}














%%%%%%%%%%%%%%%%%%%%%%%%%%%%%%%%%%%%%%%%%%%%%%%%%%%%%%%%%%%%%%%%%%%
%%%%%%%%%%%%%%%%%%%%%%%%%%%%%%%%%%%%%%%%%%%%%%%%%%%%%%%%%%%%%%%%%%%
\section{Objects, Domains, Packages, Categories}
%%%%%%%%%%%%%%%%%%%%%%%%%%%%%%%%%%%%%%%%%%%%%%%%%%%%%%%%%%%%%%%%%%%
%%%%%%%%%%%%%%%%%%%%%%%%%%%%%%%%%%%%%%%%%%%%%%%%%%%%%%%%%%%%%%%%%%%

%%%%%%%%%%%%%%%%%%%%%%%%%%%%%%%%%%%%%%%%%%%%%%%%%%%%%%%%%%%%%%%%%%%
\subsection{Object Level}
%%%%%%%%%%%%%%%%%%%%%%%%%%%%%%%%%%%%%%%%%%%%%%%%%%%%%%%%%%%%%%%%%%%
\begin{myverbatim}
> aldor -gloop
\end{myverbatim}

%%%%%%%%%%%%%%%%%%%%%%%%%%%%%%%%%%%%%%%%%%%%%%%%%%%%%%%%%%%%%%%%%%%
\begin{frame}[fragile]
  \frametitle<presentation>{Functions}
  Functions can be constructed in the following way.

\begin{myverbatim}
%3 >> &succ(n: MachineInteger): MachineInteger == n + 1
Defined succ @ (n: MachineInteger) -> MachineInteger
\end{myverbatim}

  This is (nearly) equivalent to \ldots
\begin{myverbatim}
%4 >> &pred == (n: MachineInteger): MachineInteger +-> n-1
Defined pred @ (n: MachineInteger) -> MachineInteger
\end{myverbatim}
In contrast to variable assignment (\texttt{:=}), the double equals
sign (\texttt{==}) defines constants.
\end{frame}









%%%%%%%%%%%%%%%%%%%%%%%%%%%%%%%%%%%%%%%%%%%%%%%%%%%%%%%%%%%%%%%%%%%
\begin{frame}[fragile]
  \frametitle<presentation>{Functions (cont.)}

  Functions can be assigned to variables.

\only<article>{In interpreter mode one must initialize the
  RandomNumberGenerator. If the code is compiled into an object file
  then such an initialization is done automatically.}
\begin{myverbatim}
%5 >> &import from MachineInteger, RandomNumberGenerator
%6 >> &seed(randomGenerator 0, 123456789)
123456789 @ MachineInteger

%7 >> &foo := if odd? random() then succ else pred
  () @ (n: MachineInteger) -> MachineInteger

%8 >> &foo 1
2 @ MachineInteger

%9 >> &bar := if odd? random() then succ else pred
  () @ (n: MachineInteger) -> MachineInteger

%10 >> &bar 1
0 @ MachineInteger
\end{myverbatim}
\end{frame}

















%%%%%%%%%%%%%%%%%%%%%%%%%%%%%%%%%%%%%%%%%%%%%%%%%%%%%%%%%%%%%%%%%%%
\begin{frame}[fragile]
  \frametitle<presentation>{Functions (cont.)}

  Function types must match.
\begin{myverbatim}
%11 >> &foo "abc"
       ....^
[L11 C5] #1 (Error) Argument 1 of `foo' did not match any
possible parameter type.
    The rejected type is String.
    Expected type MachineInteger.
\end{myverbatim}
\end{frame}



















%%%%%%%%%%%%%%%%%%%%%%%%%%%%%%%%%%%%%%%%%%%%%%%%%%%%%%%%%%%%%%%%%%%
\begin{frame}[fragile]
  \frametitle<presentation>{Functions (cont.)}

  Let's try concatenation of strings, given through the function
  \texttt{+}.
\begin{myverbatim}
%12 >> &"abc" + 1
       ^
[L12 C1] #1 (Error) Argument 1 of `+' did not match any
possible parameter type.
    The rejected type is String.
    Expected type MachineInteger.
\end{myverbatim}

In Aldor, \texttt{0} and \texttt{1} can be used as identifiers.
\begin{myverbatim}
% 13 >> &1:="One"
One @ String
\end{myverbatim}
\end{frame}



















%%%%%%%%%%%%%%%%%%%%%%%%%%%%%%%%%%%%%%%%%%%%%%%%%%%%%%%%%%%%%%%%%%%
\begin{frame}[fragile]
  \frametitle<presentation>{Functions (cont.)}
  Let's check what we have currently in scope.
\begin{myverbatim}
%14 >> &1
       ^
[L14 C1] #1 (Error) Have determined 2 possible types for
the expression.
        Meaning 1: MachineInteger
        Meaning 2: String
\end{myverbatim}

\begin{myverbatim}
%15 >> &+
       ^
[L15 C1] #1 (Error) Have determined 2 possible types for
the expression.
        Meaning 1: (String, String) -> String
        Meaning 2: (MachineInteger, MachineInteger)
                                    -> MachineInteger
\end{myverbatim}
\begin{myverbatim}
%16 >> &"abc" + 1
abcOne @ String
\end{myverbatim}
\end{frame}





















%%%%%%%%%%%%%%%%%%%%%%%%%%%%%%%%%%%%%%%%%%%%%%%%%%%%%%%%%%%%%%%%%%%
\begin{frame}[fragile]
  \frametitle<presentation>{Recursive Functions}

  Aldor allows to define recursive functions.
\begin{myverbatim}
%17 >> &fib(n: MachineInteger): MachineInteger == {
...    &    n=0 or n=1 => 1;
...    &    fib(n-1) + fib(n-2);
...    &  }
Defined fib @ (n: MachineInteger) -> MachineInteger

%18 >> &fib 5
8 @ MachineInteger
\end{myverbatim}
\end{frame}
\begin{myverbatim}
%19 >> &#quit
\end{myverbatim}
\begin{comment}
<<Functions>>=
aldor -gloop
succ(n: MachineInteger): MachineInteger == n + 1
pred == (n: MachineInteger): MachineInteger +-> n-1
import from MachineInteger, RandomNumberGenerator
seed(randomGenerator 0, 123456789)
foo := if odd? random() then succ else pred
foo 1
bar := if odd? random() then succ else pred
bar 1
foo "abc"
"abc" + 1
1:="One"
1
+
"abc" + 1
fib(n: MachineInteger): MachineInteger == {
	n=0 or n=1 => 1;
	fib(n-1) + fib(n-2);
}
fib 5
#quit
@
\end{comment}



















%%%%%%%%%%%%%%%%%%%%%%%%%%%%%%%%%%%%%%%%%%%%%%%%%%%%%%%%%%%%%%%%%%%
\begin{frame}[fragile]
  \frametitle<presentation>{Functions as arguments}
  Functions can be arguments to other functions.
\begin{myverbatim}
> &aldor -gloop
%3 >> &baz(
...   &    f: MachineInteger -> MachineInteger,
...   &    n: MachineInteger
...   &  ): MachineInteger == {
...   &    f f n
...   &  }
Defined baz @ (f: MachineInteger -> MachineInteger, 
n: MachineInteger) -> MachineInteger
\end{myverbatim}
\only<article>{If the interpreter detects that a line is incomplete
  after pressing ENTER, e.g. there are open parentheses that are not
  closed, then it continues the current line by showing three dots.}

\begin{myverbatim}
%4 >> &import from MachineInteger
%6 >> &baz(factorial, 3)
720 @ MachineInteger
\end{myverbatim}
\end{frame}
\begin{myverbatim}
%7 >> &#quit
\end{myverbatim}
\begin{comment}
<<Functions2>>=
aldor -gloop
baz(
  f: MachineInteger -> MachineInteger,
  n: MachineInteger
): MachineInteger == {
  f f n
}
import from MachineInteger
baz(factorial, 3)
#quit
@
\end{comment}

















%%%%%%%%%%%%%%%%%%%%%%%%%%%%%%%%%%%%%%%%%%%%%%%%%%%%%%%%%%%%%%%%%%%
\begin{frame}[fragile]
  \frametitle<presentation>{Function composition}
  It is easy to define composition of functions.
\begin{myverbatim}
%3 >> &macro I == MachineInteger;
%4 >> &import from I;

%5 >> &double(i: I): I == i+i;
%6 >> &succ(i: I): I == i+1;

%7 >> &(f: I->I) * (g: I->I): I->I == (i: I): I +-> f g i;

%8 >> &foo := double * succ;
%9 >> &bar := succ * double;

%10 >> &foo 1
4 @ MachineInteger

%11 >> &bar 1
3 @ MachineInteger
\end{myverbatim}
\end{frame}
\begin{myverbatim}
%12 >> &#quit
\end{myverbatim}
\begin{comment}
<<Functions3>>=
aldor -gloop
macro I == MachineInteger;
import from I;
double(i: I): I == i+i;
succ(i: I): I == i+1;
(f: I->I) * (g: I->I): I->I == (i: I): I +-> f g i;
foo := double * succ;
bar := succ * double;
foo 1
bar 1
#quit
@
\end{comment}















%%%%%%%%%%%%%%%%%%%%%%%%%%%%%%%%%%%%%%%%%%%%%%%%%%%%%%%%%%%%%%%%%%%
\subsection{Domains and Categories}
%%%%%%%%%%%%%%%%%%%%%%%%%%%%%%%%%%%%%%%%%%%%%%%%%%%%%%%%%%%%%%%%%%%


%%%%%%%%%%%%%%%%%%%%%%%%%%%%%%%%%%%%%%%%%%%%%%%%%%%%%%%%%%%%%%%%%%%
\begin{frame}
  \frametitle{Algebraic Structures}

  \begin{itemize}
  \item Semigroup: $(H, +)$
  \item Monoid: $(M, (+, 0))$
  \item Group: $(G, (+, -, 0))$
  \item Ring: $(R, (+, -, 0, \cdot))$
  \item Ring with 1: $(R, (+, -, 0, \cdot, 1))$
  \end{itemize}

  \begin{itemize}
  \item Universal Algebra: $(A, (f_i)_{i\in I})$ where $f_i: A^{n_i}\to A$
  \item Multisorted Algebra: $((A_j)_{j\in J}, (g_i)_{i\in I})$
  \end{itemize}
\end{frame}






%%%%%%%%%%%%%%%%%%%%%%%%%%%%%%%%%%%%%%%%%%%%%%%%%%%%%%%%%%%%%%%%%%%
\begin{frame}[fragile]
  \frametitle<presentation>{Example: Aldor Domain}
\begin{myverbatim2}
Int: with {
  &+: (%, %) -> %;&
  &0: %;&
  &1: %;&
} == add {
  Rep == Integer;
  import from Rep;
  &(x: %) + (y: %): %& == per(rep(x) + rep(y));
  &0: %& == per 0;
  &1: %& == per 1;
}
\end{myverbatim2}
\begin{itemize}
\item $\verb'Int'=(\verb'%', \textcolor{red}{(+, 0, 1)})$
\item $\verb'Rep'=((\verb'Integer',\verb'Boolean'), (+, -, \cdot, 0, 1, \ldots))$
\end{itemize}
\end{frame}














%%%%%%%%%%%%%%%%%%%%%%%%%%%%%%%%%%%%%%%%%%%%%%%%%%%%%%%%%%%%%%%%%%%
\begin{frame}[fragile]
  \frametitle<presentation>{Example: Aldor Domain}
\begin{myverbatim2}
MyInt: with {
  &+: (%, %) -> %;&
  &0: %;&
  &1: %;&
  &zero?: % -> Boolean;&
} == add {
  Rep == Integer;
  import from Rep;
  (x: %) + (y: %): % == per(rep(x) + rep(y));
  0: % == per 0;
  1: % == per 1;
  zero?(x: %): Boolean == rep(x)=0;
}
\end{myverbatim2}
\end{frame}














%%%%%%%%%%%%%%%%%%%%%%%%%%%%%%%%%%%%%%%%%%%%%%%%%%%%%%%%%%%%%%%%%%%
\begin{frame}[fragile]
  \frametitle<presentation>{Example: Extended Aldor Domain}
\begin{myverbatim}
extend MyInt: with {
  &succ: % -> %;
} == add {
  succ(x: %): % == x + 1;
}
\end{myverbatim}
\begin{itemize}
\item $((\verb'MyInt', \verb'Boolean'), ({\color{red}+, 0, 1, \verb'zero?'}))$
\item $((\verb'MyInt', \verb'Boolean'), (+, 0, 1, \verb'zero?',{\color{blue}\verb'succ'}))$
\end{itemize}
\end{frame}













%%%%%%%%%%%%%%%%%%%%%%%%%%%%%%%%%%%%%%%%%%%%%%%%%%%%%%%%%%%%%%%%%%%
\begin{comment}
%\begin{frame}[fragile]
\frametitle{Post-facto Extensions}
\begin{itemize}
\item Domains can later be extended to belong to a new category.
\item Demonstrate with \verb'Integer' in \verb'aldor -gloop' with
  libaldor and libalgebra.
%\begin{smallverbatim}
aldor -gloop                     aldor -gloop
#include "aldor"                 #include "algebra"
#include "aldorinterp"           #include "aldorinterp"
Integer has Monoid               Integer has Monoid
Integer                          Integer
#quit                            #quit
%\end{smallverbatim}
\end{itemize}
%\end{frame}
\end{comment}





















%%%%%%%%%%%%%%%%%%%%%%%%%%%%%%%%%%%%%%%%%%%%%%%%%%%%%%%%%%%%%%%%%%%
\begin{frame}[fragile]
\frametitle{Categories}
\begin{itemize}
\item A category specifies the interface of a class.

\item General syntax
\begin{myverbatim}
with {  0: %;  *: (Integer, %) -> %; }
\end{myverbatim}

\item Named categories
\begin{myverbatim}
define MyCat: Category == with { ... }
\end{myverbatim}

\item Multiple inheritance
\begin{myverbatim}
Join(CatA, CatB) with { ... }
\end{myverbatim}
\end{itemize}
\end{frame}












%%%%%%%%%%%%%%%%%%%%%%%%%%%%%%%%%%%%%%%%%%%%%%%%%%%%%%%%%%%%%%%%%%%
\begin{frame}[fragile]
\frametitle{Domains}
\begin{itemize}
\item A domain realizes the interface given through a category.
\item General syntax
\begin{myverbatim}
add {
  Rep == Record(real: Integer, imag: Integer);
  import from Rep, Integer;
  0: % == per [0, 0];
  re(a: %): Integer == rep(a).real;
  im(a: %): Integer == rep(a).imag;
}
\end{myverbatim}
\end{itemize}
\end{frame}



















%%%%%%%%%%%%%%%%%%%%%%%%%%%%%%%%%%%%%%%%%%%%%%%%%%%%%%%%%%%%%%%%%%%
\begin{frame}[fragile]
\frametitle{Categories + Domains}
\begin{myverbatim}
define ComplexCategory: Category == with {
  0: %;
  re: % -> Integer;
}

ComplexDomain: ComplexCategory == add {
  Rep == Record(real: Integer, imag: Integer);
  import from Rep, Integer;
  0: % == per [0, 0];
  re(a: %): Integer == rep(a).real;
  &im(a: %): Integer == rep(a).imag;
}
\end{myverbatim}
\end{frame}



















%%%%%%%%%%%%%%%%%%%%%%%%%%%%%%%%%%%%%%%%%%%%%%%%%%%%%%%%%%%%%%%%%%%
\begin{frame}[fragile]
\frametitle{Inheritance and Data Hiding}
\begin{myverbatim}
define ComplexCategory: Category == with {
  0: %;
  re: % -> Integer;
}

MyComplex: ComplexCategory == Integer add {
  Rep == Integer;
  import from Rep;
  re(a: %): Integer == rep(a);
}  
\end{myverbatim}
\end{frame}



















%%%%%%%%%%%%%%%%%%%%%%%%%%%%%%%%%%%%%%%%%%%%%%%%%%%%%%%%%%%%%%%%%%%
%\begin{comment}
%\begin{frame}[fragile]
\frametitle{Packages}
\begin{itemize}
\item A package is a domain that has no \verb'%'.
\begin{myverbatim}
PoweringPackage(R: Monoid): with {
  binaryExponentiation: (R, Integer) -> R
} == add {
  binaryExponentiation(x: R, n: Integer): R == {
    ...
  }
}
\end{myverbatim}
\end{itemize}
%\end{frame}
%\end{comment}











%%%%%%%%%%%%%%%%%%%%%%%%%%%%%%%%%%%%%%%%%%%%%%%%%%%%%%%%%%%%%%%%%%%
\begin{frame}[fragile]
\frametitle{Default Implementations}
\begin{myverbatim}
define PrimitiveType: Category == with {
    =: (%, %) -> Boolean;
    ~=: (%, %) -> Boolean;
  &default {
  &  (a:%) ~= (b:%):Boolean == ~(a = b);
  &}
}
\end{myverbatim}
\end{frame}


















%%%%%%%%%%%%%%%%%%%%%%%%%%%%%%%%%%%%%%%%%%%%%%%%%%%%%%%%%%%%%%%%%%%
\begin{frame}[fragile]
\frametitle{Conditional Exports}

\begin{myverbatim}
Partial(T:Type): with {
  &if T has PrimitiveType then PrimitiveType;
  ...
} == add {
  ...
  &if T has PrimitiveType then {
    (x:%) = (y:%):Boolean == {
      import from T;
      fy? := failed? y;
      failed? x => fy?;
      ~fy? and retract x = retract y;
    }
  &}
}
\end{myverbatim}
\end{frame}




























%%%%%%%%%%%%%%%%%%%%%%%%%%%%%%%%%%%%%%%%%%%%%%%%%%%%%%%%%%%%%%%%%%%
\begin{frame}[fragile]
\frametitle{Domain-dependent (higher order) Functions}

\begin{itemize}
\item \alert{type} of result/input depends on \alert{value} of input
\item $(a: A) \times B(a) \to C(a)$
\item identity: (n: Integer, R: Ring) -> Matrix(n, n, R)
\end{itemize}
\begin{myverbatim}
#include "algebra"
#include "aldorio"
mygcd(R: EuclideanDomain)(a: R, b: R): R == {
  R has Field => if zero? a and zero? b then 0 else 1;
  while not zero? b repeat { (a, b) := (b, a rem b) }
  return a;
}
Q ==> Fraction Integer;
import from Integer, Q;
g := mygcd Integer;
stdout << g(4,6) << newline;
stdout << mygcd(Q)(4::Q,6::Q) << newline;
\end{myverbatim}
\end{frame}










%%%%%%%%%%%%%%%%%%%%%%%%%%%%%%%%%%%%%%%%%%%%%%%%%%%%%%%%%%%%%%%%%%%
\begin{frame}[fragile]
\frametitle{Recursively Defined Domain-valued Functions}

Example from Aldor-Combinat:

\begin{myverbatim2}
&T&(L: LT): CS L == (&X + T*T&)(L) add;
egs := generatingSeries $ T(Integer);
n := count(egs, 5);
\end{myverbatim2}%$

Can define mutually recursive types
\begin{myverbatim2}
&A&(L: LT): CS L == (&E + X*B*B*B&)(L) add;
&B&(L: LT): CS L == (&E + X*A*A&)(L) add;
ogs: isomorphismTypeGeneratingSeries $ A(String);
m := count(ogs, 6);
\end{myverbatim2}%$


\end{frame}




























%%%%%%%%%%%%%%%%%%%%%%%%%%%%%%%%%%%%%%%%%%%%%%%%%%%%%%%%%%%%%%%%%%%
\section{Collections, Iterations, Generators, Streams}
%%%%%%%%%%%%%%%%%%%%%%%%%%%%%%%%%%%%%%%%%%%%%%%%%%%%%%%%%%%%%%%%%%%
%%%%%%%%%%%%%%%%%%%%%%%%%%%%%%%%%%%%%%%%%%%%%%%%%%%%%%%%%%%%%%%%%%%
\subsection{Collections}
%%%%%%%%%%%%%%%%%%%%%%%%%%%%%%%%%%%%%%%%%%%%%%%%%%%%%%%%%%%%%%%%%%%

There are already some collections implemented in the Aldor library,
for example, 


%%%%%%%%%%%%%%%%%%%%%%%%%%%%%%%%%%%%%%%%%%%%%%%%%%%%%%%%%%%%%%%%%%%
\begin{frame}[fragile]
  \frametitle<presentation>{Collections}
\begin{itemize}
\item {List}, SortedList
\item {Array}, PrimitiveArray,
\item {HashTable}, SortedAssociationSet,
\item Set, SortedSet,
\item etc.
\end{itemize}
\end{frame}














Lists can be created in different ways.
\begin{myverbatim}
%3 >> macro I == MachineInteger;
%4 >> import from I, List I;
\end{myverbatim}

%%%%%%%%%%%%%%%%%%%%%%%%%%%%%%%%%%%%%%%%%%%%%%%%%%%%%%%%%%%%%%%%%%%
\begin{frame}[fragile]
  \frametitle<presentation>{Collections (cont.)}

Lists can be given by their elements.
\begin{myverbatim}
%5 >> &u1 := [2,3,5,7,11,13,17,19]
[2,3,5,7,11,13,17,19] @ List(MachineInteger)
\end{myverbatim}
They can also be given through iterators.
\begin{myverbatim}
%6 >> &u2 := [2*i for i in 1..4]
[2,4,6,8] @ List(MachineInteger)
\end{myverbatim}
Counting downwards is also OK.
\begin{myverbatim}
%7 >> &u3 := [2*i for i in 9 .. -9 by -2]
[18,14,10,6,2,-2,-6,-10,-14,-18] @ List(MachineInteger)
\end{myverbatim}
Additionally one can include a ``such that'' expression which includes
the value only if the condition is true.
\begin{myverbatim}
%8 >> &u4 := [5*i for i in 20 .. -20 by -3 | bit?(i,2)]
[100,70,25,-5,-20,-50,-95] @ List(MachineInteger)
\end{myverbatim}
\end{frame}














%%%%%%%%%%%%%%%%%%%%%%%%%%%%%%%%%%%%%%%%%%%%%%%%%%%%%%%%%%%%%%%%%%%
\begin{frame}[fragile]
  \frametitle<presentation>{Collections (cont.)}

One can iterate over lists.
\begin{myverbatim}
%9 >> &[3*i for i in u1 | not(i=11)]
[6,9,15,21,39,51,57] @ List(MachineInteger)
\end{myverbatim}
Or access list elements through their index.
\begin{myverbatim}
%10 >> &[u3.(2*i) for i in 1..3]
[14,6,-2] @ List(MachineInteger)
\end{myverbatim}
Go in parallel through two lists.
\begin{myverbatim}
%11 >> &[a+b for a in [1, 4, 0] for b in [2,5,7,9,11]]
[3,9,7] @ List(MachineInteger)
\end{myverbatim}
Note that the result is as long as the shortest list.
\end{frame}






%%%%%%%%%%%%%%%%%%%%%%%%%%%%%%%%%%%%%%%%%%%%%%%%%%%%%%%%%%%%%%%%%%%
\begin{frame}[fragile]
  \frametitle<presentation>{Collections (cont.)}

A list of lists is also OK.
\begin{myverbatim}
%12 >> &v1: List List I := [u1,u2]
[[2,3,5,7,11,13,17,19],[2,4,6,8]] @ List(List(MachineInteger))
\end{myverbatim}
One can also have arrays of lists.
\begin{myverbatim}
%13 >> &a1: Array List I := [u1,u2]
[[2,3,5,7,11,13,17,19],[2,4,6,8]] @ Array(List(MachineInteger))
\end{myverbatim}
Note that elements in the list must have the same type.
\end{frame}
\begin{myverbatim}
%14 >> &#quit
\end{myverbatim}
\begin{comment}
<<Collections>>=
aldor -gloop
macro I == MachineInteger;
import from I, List I;
u1 := [2,3,5,7,11,13,17,19]
u2 := [2*i for i in 1..4]
u3 := [2*i for i in 9 .. -9 by -2]
u4 := [5*i for i in 20 .. -20 by -3 | bit?(i,2)]
[3*i for i in u1 | not(i=11)]
[u3.(2*i) for i in 1..3]
[a+b for a in [1, 4, 0] for b in [2,5,7,9,11]]
v1: List List I := [u1,u2]
a1: Array List I := [u1,u2]
#quit
@
\end{comment}

















%%%%%%%%%%%%%%%%%%%%%%%%%%%%%%%%%%%%%%%%%%%%%%%%%%%%%%%%%%%%%%%%%%%
\subsection{Iterations}
%%%%%%%%%%%%%%%%%%%%%%%%%%%%%%%%%%%%%%%%%%%%%%%%%%%%%%%%%%%%%%%%%%%
\begin{myverbatim}
%3 >> macro I == MachineInteger;
%4 >> import from I, List I;
\end{myverbatim}

%%%%%%%%%%%%%%%%%%%%%%%%%%%%%%%%%%%%%%%%%%%%%%%%%%%%%%%%%%%%%%%%%%%
\begin{frame}[fragile]
  \frametitle<presentation>{Loops}

The general form of a loop is
\begin{center}
  \textit{Iterators} \texttt{repeat} \textit{Body}
\end{center}
Iterators could be \texttt{for} expressions \ldots
\begin{myverbatim}
%5 >> &for i in 3..6 repeat stdout << i << space
3 4 5 6
\end{myverbatim}
\texttt{while} expressions \ldots
\begin{myverbatim}
%6 >> &k:=7; while k<20 repeat {k:=k+3; stdout << k << space}
10 13 16 19 22
\end{myverbatim}
or a combination of them.
\end{frame}
\begin{myverbatim}
%7 >> &for i in 1..20 | odd? i while i < 15 repeat stdout << i << space
1 3 5 7 9 11 13
%8 >> &#quit
\end{myverbatim}
\begin{comment}
<<Collections>>=
aldor -gloop
macro I == MachineInteger;
import from I, List I;
for i in 3..6 repeat stdout << i << space
k := 7; while k < 20 repeat {k := k+3; stdout << k << space}
for i in 1..20 | odd? i while i < 15 repeat stdout << i << space
#quit
@
\end{comment}

































%%%%%%%%%%%%%%%%%%%%%%%%%%%%%%%%%%%%%%%%%%%%%%%%%%%%%%%%%%%%%%%%%%%
\subsection{Generators}
%%%%%%%%%%%%%%%%%%%%%%%%%%%%%%%%%%%%%%%%%%%%%%%%%%%%%%%%%%%%%%%%%%%

Through the concept of a \texttt{Generator}, Aldor allows a uniform
treatment of different data structures.
\begin{myverbatim}
%3 >> macro I == MachineInteger;
%4 >> import from I, List I, Array I, HashTable(I, I);
\end{myverbatim}

%%%%%%%%%%%%%%%%%%%%%%%%%%%%%%%%%%%%%%%%%%%%%%%%%%%%%%%%%%%%%%%%%%%
\begin{frame}[fragile]
  \frametitle<presentation>{Generator (Setup of data structures)}

\begin{myverbatim}
%5 >> &l: List I  := [2,5,7,22];
[2,5,7,22] @ List(MachineInteger)

%6 >> &a: Array I := [3,2];
[3,2] @ Array(MachineInteger)

%7 >> &h: HashTable(I, I) := table();
[8] @ HashTable(MachineInteger, MachineInteger)

%8 >> &h.4 := 3;
3 @ MachineInteger

%9 >> &h.(-88) := -2;
-2 @ MachineInteger
\end{myverbatim}
\end{frame}






%%%%%%%%%%%%%%%%%%%%%%%%%%%%%%%%%%%%%%%%%%%%%%%%%%%%%%%%%%%%%%%%%%%
\begin{frame}[fragile]
  \frametitle<presentation>{Generator (Uniform treatment)}

\begin{myverbatim}
%10 >> &for i in l         repeat stdout << i << space
2 5 7 22

%11 >> &for i in a         repeat stdout << i << space
3 2

%12 >> &for i in keys h    repeat stdout << i << space
-88 4

%13 >> &for i in entries h repeat stdout << i << space
-2 3
\end{myverbatim}
\end{frame}
\begin{comment}
<<Generator>>=
aldor -gloop
macro I == MachineInteger;
import from I;
l: List I  := [2,5,7,22];
a: Array I := [3,2];
h: HashTable(I, I) := table();
h.4 := 3;
h.(-88) := -2;
for i in l         repeat stdout << i << space
for i in a         repeat stdout << i << space
for i in keys h    repeat stdout << i << space
for i in entries h repeat stdout << i << space
@
\end{comment}








%%%%%%%%%%%%%%%%%%%%%%%%%%%%%%%%%%%%%%%%%%%%%%%%%%%%%%%%%%%%%%%%%%%
\begin{frame}[fragile]
  \frametitle<presentation>{Generator}

Generators are normally used in \texttt{for} constructs.
\begin{center}
  \texttt{for} \textit{variable} \texttt{in} \textit{generator}
\end{center}

Generators can be seen as functions that yield a new (the next)
value.

\only<article>{They are created in the following way.}
\begin{myverbatim}
%14 >> &g1: Generator I := generate {yield 4; yield 2}
%15 >> &g2: Generator I := generate {for i in g1 repeat {
...    &     yield i; yield 0
...    & }}
%16 >> &for i in g2 repeat stdout << i << space
4 0 2 0
\end{myverbatim}

\only<article>{And can be used like \ldots}
\begin{myverbatim}
%16 >> &for i in g2 repeat stdout << i << space
4 0 2 0
\end{myverbatim}
\end{frame}
Of course, there can be any program logic inside a \texttt{generate}
expression. Evaluation stops temporarily if a \texttt{yield} statement
is reached and is resumed at that place if the generator is asked for
the next value.

As anything else in Aldor, Generators are \emph{first class} values,
i.e., they can be used as arguments to functions.
For example, one could write a function that takes a function and a
generator and returns a generator that generates elements where the
given function is applied.
\begin{myverbatim}
%17 >> &map(f: I -> List I, g: Generator I): Generator List I == {
...    &   f i for i in g  
...    & }

%18 >> &f1(n: I): List I == [n]
%19 >> &f2(n: I): List I == [0 for i in 1..n]
%20 >> &ll: List List I := [map(f1, g1)]
[] @ List(List(MachineInteger))
\end{myverbatim}
Why do we get an empty list? We have already exhausted $g_1$ above by
generating all elements of $g_2$. So there are no \emph{next} elements
anymore.
\begin{myverbatim}
%21 >> [map(f1, generator l)]
[[2],[5],[7],[22]] @ List(List(MachineInteger))

%22 >> [map(f2, generator a)]
[[0,0,0],[0,0]] @ List(List(MachineInteger))
\end{myverbatim}


\begin{comment}
<<Generator>>=
g1: Generator I := generate {yield 4; yield 2}
g2: Generator I := generate {for i in g1 repeat {
    yield i; yield 0
}}
for i in g2 repeat stdout << i << space
map(f: I -> List I, g: Generator I): Generator List I == {
  f i for i in g  
}
f1(n: I): List I == [n]
f2(n: I): List I == [0 for i in 1..n]
ll: List List I := [map(f1, g1)]
[map(f1, generator l)]
[map(f2, generator a)]
@
\end{comment}



















%%%%%%%%%%%%%%%%%%%%%%%%%%%%%%%%%%%%%%%%%%%%%%%%%%%%%%%%%%%%%%%%%%%
\begin{frame}[fragile]
  \frametitle<presentation>{Infinite Generators}

A generator can generate infinitely many objects.
\begin{myverbatim}
%23 >> &g1 := generate repeat {yield 0; yield 1}

%24 >> &for k in 1..5 for i in g1 repeat stdout << i << space
0 1 0 1 0

%25 >> &for k in l    for i in g1 repeat stdout << i << space
1 0 1 0
\end{myverbatim}
\end{frame}
\begin{comment}
<<Generator>>=
g1 := generate repeat {yield 0; yield 1}
for k in 1..5 for i in g1 repeat stdout << i << space
for k in l    for i in g1 repeat stdout << i << space
#quit
@
\end{comment}































%%%%%%%%%%%%%%%%%%%%%%%%%%%%%%%%%%%%%%%%%%%%%%%%%%%%%%%%%%%%%%%%%%%
\subsection{Streams}
%%%%%%%%%%%%%%%%%%%%%%%%%%%%%%%%%%%%%%%%%%%%%%%%%%%%%%%%%%%%%%%%%%%


%%%%%%%%%%%%%%%%%%%%%%%%%%%%%%%%%%%%%%%%%%%%%%%%%%%%%%%%%%%%%%%%%%%
\begin{frame}[fragile]
  \frametitle<presentation>{Stream}

The Aldor library provides a container for infinite objects.
\begin{myverbatim}
%26 >> &s: Stream I := [g1]
[...] @ Stream(MachineInteger)
\end{myverbatim}
Streams are lazy, i.e. they only compute values when values are
needed.
\begin{myverbatim}
%26 >> &s.3
0 @ MachineInteger

%28 >> &s
[1,0,1,0,...] @ Stream(MachineInteger)
\end{myverbatim}
Streams are like generators with memory.
\end{frame}












%%%%%%%%%%%%%%%%%%%%%%%%%%%%%%%%%%%%%%%%%%%%%%%%%%%%%%%%%%%%%%%%%%%
\begin{frame}[fragile]
  \frametitle<presentation>{Stream (cont.)}

Of course, one can create functions that take such \emph{infinite}
objects as arguments.
\begin{myverbatim}
%29 >> &fib(s: Stream I): Stream I == {
...    &   stream( (i: I, t: Stream I): I +-> {
...    &     i=0 or i=1 => s.i;
...    &     t.(i-1) + t.(i-2) + s.i;
...    &   })
...    & }

%30 >> &t := fib s
[...] @ Stream(MachineInteger)
%31 >> &t.9
54 @ MachineInteger
%32 >> &t
[1,0,2,2,5,7,13,20,34,54,...] @ Stream(MachineInteger)
\end{myverbatim}
\end{frame}
\begin{myverbatim}
%14 >> &#quit
\end{myverbatim}
\begin{comment}
<<Stream>>=
fib(s: Stream I): Stream I == {
  stream( (i: I, t: Stream I): I +-> {
    i=0 or i=1 => s.i;
    t.(i-1) + t.(i-2) + s.i;
  })
}
t := fib s
t.9
t
#quit
@
\end{comment}















%%%%%%%%%%%%%%%%%%%%%%%%%%%%%%%%%%%%%%%%%%%%%%%%%%%%%%%%%%%%%%%%%%%
\begin{frame}<presentation>
  \frametitle{The End}
  \begin{center}
    \LARGE Thank you.
  \end{center}
\end{frame}











%%%%%%%%%%%%%%%%%%%%%%%%%%%%%%%%%%%%%%%%%%%%%%%%%%%%%%%%%%%%%%%%%%%
%%%%%%%%%%%%%%%%%%%%%%%%%%%%%%%%%%%%%%%%%%%%%%%%%%%%%%%%%%%%%%%%%%%
%%%%%%%%%%%%%%%%%%%%%%%%%%%%%%%%%%%%%%%%%%%%%%%%%%%%%%%%%%%%%%%%%%%
%%%%%%%%%%%%%%%%%%%%%%%%%%%%%%%%%%%%%%%%%%%%%%%%%%%%%%%%%%%%%%%%%%%
\end{document}
%%%%%%%%%%%%%%%%%%%%%%%%%%%%%%%%%%%%%%%%%%%%%%%%%%%%%%%%%%%%%%%%%%%
%%%%%%%%%%%%%%%%%%%%%%%%%%%%%%%%%%%%%%%%%%%%%%%%%%%%%%%%%%%%%%%%%%%
%%%%%%%%%%%%%%%%%%%%%%%%%%%%%%%%%%%%%%%%%%%%%%%%%%%%%%%%%%%%%%%%%%%
%%%%%%%%%%%%%%%%%%%%%%%%%%%%%%%%%%%%%%%%%%%%%%%%%%%%%%%%%%%%%%%%%%%
